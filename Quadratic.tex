\documentclass{article}
\usepackage{graphicx} % Required for inserting images
\usepackage{graphicx, amsmath, amsthm, amssymb}
\usepackage{url}
\usepackage[utf8]{inputenc}
%\usepackage{natbib}
\usepackage{verbatim}
\usepackage{hyperref}

\hypersetup{ 
    % change default hyperrefs set up, such as removing boxes around the link, 
    % or changing the color of the links
    colorlinks=true,
    linkcolor=blue,
    urlcolor=cyan,
    citecolor=red,
    filecolor=magenta,     
}

\title{Proof Of Quadratic Formula}
\author{Ethan Denning}
\date{September 22, 2025}

\theoremstyle{plain}
\newtheorem{thm}{Theorem}

\theoremstyle{definition}
\newtheorem{defi}[thm]{Definition}
\newtheorem{rem}[thm]{Remark}

\begin{document}   
\maketitle

%MAKE NOTE ON ASSUME A IS NON ZERO

\section{Introduction}
In this short paper we will show that the Quadratic Formula can be derived from completing the square of a Quadratic Equation. In section \ref{sect:prelims} we will introduce what a quadratic equation is, the idea and process of completing the square of a quadratic equation, and the quadratic quadratic formula. In section \ref{sect:derive} we will derive the Quadratic Formula. Finally in section \ref{sect:application} we will discuss applications of the Quadratic Formula and the importance of it within mathematics.\\\\
It is additionally worth mentioning that throughout this paper we will be working over the set of real numbers with quadratic equations that only yield real solutions. 

\section{Preliminaries} \label{sect:prelims}
First we will begin by defining a quadratic equation.
    \begin{defi}\label{def:quadratic}
    A quadratic equation is a polynomial of degree two which can be represented in the standard form\hfill\\
    $$ax^2+bx+c=0$$
    The variable $x$ represents an unknown number belonging to the set of real numbers, and $a, b, $ and $c$ represent known numbers that belong to the set of real numbers.\\
    Additionally, the values of $x$ that satisfy the quadratic equation are known as the equation's roots or zeroes. For the sake of working specifically over the set of real numbers a quadratic equation can have at most two real roots.\\
    \end{defi}
    \begin{defi}\label{def:compSquare}
    Completing the square of a quadratic equation is the process of taking a quadratic polynomial in standard form,\\
    $$ax^2+bc+c=0$$
    and converting it to form,
    $$a(x-h)^2+k$$
    where $h$ and $k$ belong to the set of real numbers.
    The process of completing the square can be broken up into elementary steps as such.\\
    Given standard form quadratic equation, $$ax^2+bx+c=0$$
    divide the equation by a so that $x$ no longer has a coefficient, $$x^2+\frac{bx}{a}+\frac{c}{a}=0$$
    Subtracting $\frac{c}{a}$ from both sides of the equations yields, 
    $$x^2+\frac{bx}{a}=-\frac{c}{a}$$
    Additionally we will add $(\frac{b}{2a})^2$ to both sides of the equation to give us,
    $$x^2+\frac{bx}{a}+(\frac{b}{2a})^2= -(\frac{c}{a}) + (\frac{b}{2a})^2$$
    The left side of the equation can be reduced to a perfect square, and moving the terms from the right side of the equation to the left of the equation yields, 
    $$(x+\frac{b}{2a})^2+\frac{c}{a}-\frac{b^2}{4a^2}=0$$
    Notice that we have an equation in the form of $a(x-h)^2+k$, where $h = -\frac{b}{2a}$ and $k=\frac{c}{a} - \frac{b^2}{4a^2}$.
    \end{defi}
    \begin{defi}\label{def:quadraticEq}
    To find the roots of a standard form quadratic equation you can use the quadratic formula which is in the form
    $$x=\frac{-b\pm\sqrt{b^2-4ac}}{2a}$$
    \end{defi}

\section{Derivation}    \label{sect:derive}
    \begin{proof}
    
    Let $ax^2+bx+c=0$ be a quadratic equation in standard form, where x is a variable of unknown roots belonging to the set of real numbers, and $a,b,c$ are known coefficients that belong to the set of real numbers.\\
    We will begin the process of completing the square of the standard polynomial, as seen in section \ref{sect:prelims}, up to the point where we arrive at,
    $$x^2+\frac{bx}{a}+(\frac{b}{2a})^2=-(\frac{c}{a})+(\frac{b}{2a})^2$$
    The left side of the equation can be reduced to,
    $$(x+\frac{b}{2a})^2=-(\frac{c}{a})+\frac{b^2}{4a^2}$$
    By taking the square root of the left and right side of the equation you are left with
    $$x+\frac{b}{2a}=\pm\sqrt{-(\frac{c}{a})+\frac{b^2}{4a^2}}$$
    Through cross multiplication of the terms within the radical sign on the right side of the equation you yield,
    $$x+\frac{b}{2a}=\pm\sqrt{\frac{-4a^2c+ab^2}{4a^3}}$$
    Notice that the right side of the equation above can factor out an $a$,
    $$x+\frac{b}{2a}=\pm\sqrt{\frac{-4ac+b^2}{4a^2}}$$
    Additionally notice that the denominator of the right side of the equation is $\sqrt{4a^2}$, which can be further simplified,
    $$x+\frac{b}{2a}={\pm\frac{\sqrt{-4ac+b^2}}{2a}}$$
    Subtracting $\frac{b}{2a}$ from both sides of the equation to isolate $a$, and noticing the like denominators of 2a on the right side finally yields,
     $$x=\frac{-b\pm\sqrt{b^2-4ac}}{2a}$$
    \end{proof}
\section{Applications}  \label{sect:application}

\begin{thebibliography}{100}
       
\end{thebibliography}
\end{document}
