\documentclass[12pt]{article}

\usepackage{graphicx, amsmath, amsthm, amssymb}
\usepackage{url}
\usepackage[utf8]{inputenc}
%\usepackage{natbib}
\usepackage{verbatim}
\usepackage{hyperref}

\hypersetup{ 
    % change default hyperrefs set up, such as removing boxes around the link, 
    % or changing the color of the links
    colorlinks=true,
    linkcolor=blue,
    urlcolor=cyan,
    citecolor=red,
    filecolor=magenta,     
}

\title{The Irrationality of $\sqrt {2}$ }

\author{Ethan Denning}

\date{September 2, 2025}

\theoremstyle{plain}
\newtheorem{thm}{Theorem}

\theoremstyle{definition}
\newtheorem{defi}[thm]{Definition}
\newtheorem{rem}[thm]{Remark}

\begin{document}

\maketitle

    \section{Introduction}
    
    In this short paper we will show that $\sqrt{2}$ is an irrational number. In Section \ref{sect:prelim} we will introduce the concepts of rational and irrational numbers. In Section \ref{sect:proof},
    
    % the command \ref makes hyperlink around the number of the Section and you can click on it in a pdf file that will jump to the section indicated.
    % simply use \ref{<label>} where <label> is the section/equation that you tag and want reader to focus on
    
    we will provide a proof of the fact that $\sqrt{2}$ is irrational. Finally, in \ref{sect:irrational}, we will discuss what we know about other numbers that are known to be irrational. 

    \section{Preliminaries} \label{sect:prelim}

    First, we define the rational numbers.

    \begin{defi}\label{def:rational}
        A \emph{rational number} is a real number that can be expressed in the form $\dfrac{m}{n}$ where $m$ and $n$ are integers, and $n\ne 0$. \\
        For example rational numbers such as $5$ or $9.2$ can be expressed as $\frac{10}{2}$ and $\frac{92}{10}$ respectively. 
    \end{defi}

    \begin{defi}\label{defi:irrationality}
        An \emph{irrational number} is any real number that cannot be expressed as the quotient of two integers. Irrational numbers such as $\pi$ can not be expressed as the ratio of two integers. If expressed as a decimal, $\pi$ would be infinitely expanding with no regularly repeating groups of digits. 
    \end{defi}
    
    \section{The Proof} \label{sect:proof}

    \begin{thm}\label{thm:main}
        The real number $\sqrt{2}$ is irrational.
    \end{thm}
    
    \begin{proof}
    [{Proof of Theorem \ref{thm:main}}]\hfill\\
    
    We will prove the statement by contradiction. First, let $\alpha=\sqrt{2}$ and, for the sake of contradiction, we assume that $\alpha$ is a rational number in lowest terms. This is because any rational number can be represented uniquely in lowest terms, and doing so prevents us from initially having a common factor which is relied upon later within the proof.\hfill\\
    Then, we can write $\alpha=\frac{m}{n}$, where  $m$ and $q$ are integer numbers and have no common factors other than 1 (in lowest terms).\\

    %Complete the proof here. There will be the QED symbol at the end.
    % For large fraction and you want the brackets to extend or increase in size to match the height of the fraction, use \left( and \right) commands.
    Where
    $$
        %\left(\dfrac{\dfrac{m}{n}}{\dfrac{a}{b}}\right)^2=
        \alpha^2 = \left(\dfrac{m}{n}\right)^2 = 2
    $$
    Since $\left(\dfrac{m}{n}\right)^2 = 2$, by isolating $m$ it is shown that \\
    $$
    m^2 = 2n^2.
    $$\\
    The expression $m^2 = 2n^2$ implies that the value of $m^2$ must be an even number since it is equal to 2 multiplied by some number $q^2$. Since $m^2$ is even, it is additionally implied that $m$ is even and can be divided by 2.\\
    The statement "$m$ is even" while not trivial, can be proven with the contrapositive. \cite{Chili Math}\\\\
    The number $m$ can now be represented as
    $$m = 2m'$$
    where $m'$ is some other whole number.\\ \\
    Substituting $m$ = $2m'$ into equation $m^2 = 2n^2$, it is observed that
    $$2m'^2 = n^2$$
    Repeating this process now for equation $n^2 = 2m^2$ can make the observation that both $m$ and $n$ share a common factor of 2, thus implying a contradiction.\\

    \end{proof}

    \section{Irrational Numbers} \label{sect:irrational}
    
    Proof by contradiction is one of several ways to prove the irrationality of $\sqrt2$. For example, an additional analytic method used to prove $\sqrt2$'s irrationality is via $Bezout's$ $Lemma$.
    \cite{Alexander Bogomolny}. \\\\
    The set of irrational numbers is a higher order of infinity to that of rational numbers, and among these infinite irrationals the irrational number $\pi$ seems to be dealt with most in intermediate mathematics.\\
    While it is possible to prove $\pi's$ irrationality via contradiction \cite{Wikipedia}, it can be helpful to couple the proper proof with this visual \href{https://www.youtube.com/watch?v=YsIlEVeHc4k}{(see here)}.The visualization gives an intuitive understanding of irrationality by giving a visual of infinite non-repeating decimal expansion. If $\pi$ were rational there would be no infinite pattern, instead the same pattern would repeat over and over.\\\\
    It is additionally worth mentioning that the above proof by contradiction method used to prove the irrationality of $\sqrt{2}$ will not work to prove the irrationality of $\pi$.\\
    Proof by contradiction of $\sqrt{2}$ works because $\sqrt{2}$ is a solution of a quadratic polynomial for example $x^2-2=0$. However, $\pi$ is transcendental\cite{Nasa}, meaning that there is no polynomial equation that it can satisfy. 
    
    %\bibliographystyle{plainnat}
    %\bibliography{refs}

\begin{thebibliography}{100}
        \bibitem{Alexander Bogomolny} Alexander Bogomolny. ``Cut-The-Knot''. Published 2020.\\ Link to source: \href{https://www.cut-the-knot.org/proofs/FloydSqRt.shtml}{(see here)}
        \bibitem{Wikipedia} Wikipedia. "Irrationality of pi". Last Modified 21 June 2025.\\ 
        Link to source: \href{https://en.wikipedia.org/wiki/Proof_that_%CF%80_is_irrational}{(see here)}
        \bibitem{Chili Math} Chili Math. "Prove: Suppose n is an integer. If $n^2$ is even, then $n$ is even.". Date of access 10 September 2025.\\ Link to source: \href{https://www.chilimath.com/lessons/basic-math-proofs/if-n-squared-is-even-then-n-is-even/}{(see  here)}
        \bibitem{Nasa} Nasa.gov. "The Transcendentality of $\pi$." Date of access 10 September 2025.\\
        Link to source: \href{https://www.grc.nasa.gov/www/k-12/Numbers/Math/Mathematical_Thinking/transcendentality_of_p.htm}{(see here)}
    \end{thebibliography}
\end{document}
