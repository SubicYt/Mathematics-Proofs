\documentclass{article}
\usepackage{graphicx} % Required for inserting images
\usepackage{graphicx, amsmath, amsthm, amssymb}
\usepackage{xcolor}
\usepackage{url}
\usepackage[utf8]{inputenc}
%\usepackage{natbib}
\usepackage{verbatim}
\usepackage{hyperref}
\usepackage{float} 

\hypersetup{ 
    % change default hyperrefs set up, such as removing boxes around the link, 
    % or changing the color of the links
    colorlinks=true,
    linkcolor=blue,
    urlcolor=cyan,
    citecolor=red,
    filecolor=magenta,     
}

\newenvironment{example}
\theoremstyle{plain}
\newtheorem{thm}{Theorem}
\theoremstyle{definition}
\newtheorem{defi}[thm]{Definition}
\newtheorem{rem}[thm]{Remark}
\newtheorem{lemma}{Lemma} 



\title{The Mean Value Theorem}
\author{Ethan Denning}
\date{October 2025}

\begin{document}

\maketitle

\section{Introduction}
In this paper we will prove introduce and prove the Mean Value Theorem using Rolle's Theorem. In section \ref{sect:prelim} we will introduce the concept of Rolle's theorem as well as the mean value theorem. In section \ref{sect:lemma} we will further explore Rolle's theorem and prove it. Finally in section \ref{sect:MVT} we will go on to prove the Mean Value Theorem using Rolle's theorem. 

\section{Preliminaries} \label{sect:prelim}
First we will begin by introducing Rolle's Theorem.
    \begin{thm} \label{rolleThm}
        \emph{Rolle's Theorem states that any differentiable real valued function that has equal values at two points must have a tangent line of zero to the slope in between those two points.} \\\\
        Let f be a function that satisfies the following conditions.
        \begin{enumerate}
            \item f is continuous on the closed interval $(a,b)$
            \item f is differentiable on the open interval $[a,b]$
            \item $f(a)$ = $f(b)$
        \end{enumerate}
        Then there must be some point c within the closed interval (a,b) where $f'(c) = 0$ \href{fig:rolles}{See Figure 1}
    \begin{figure}[H]
        \centering
        \includegraphics[width=0.5\linewidth]{rollesThm.png}
        \caption{Green Tangent Line, Red Secant Line}
        \label{fig:rolles}
    \end{figure}
    
    \end{thm}
Next we will go on to introduce the Mean Value Theorem.
    \begin{thm} \label{meanThm}
        \emph{The Mean Value Theorem is a more generalized form of Rolle's Theorem that states that for any differentiable real valued function with two endpoints, there is a point where the slope of tangent line along the function is equal to the slope of the secant line between the two endpoints.}\\\\
    Let f be a function that satisfies the following conditions.
    \begin{enumerate}
        \item f is continuous on the closed interval $[a,b]$
        \item f is differentiable on the the open interval $(a,b)$, where $a < b$
    \end{enumerate}
    Then there exists some point $c$ within $(a,b)$ such that 
    $$f'(c) = \frac{f(b)-f(a)}{b-a}$$.
    \href{fig:MVT}{See figure 2}
    \begin{figure}[H]
        \centering
        \includegraphics[width=0.5\linewidth]{meanvalue.png}
        \caption{Notice a tangent line at point c parallel to the secant line.}
        \label{fig:MVT}
    \end{figure}
    \end{thm}
\section{Proof Of Theorem 1} \label{sect:lemma}
    \begin{proof}
        When proving Rolle's Theorem there are three cases that must be taken into account. 
        \begin{enumerate}
            \item $f(x) = K$ for some $K$ being a constant
            \item $f(x) > f(a)$ for some $x$ within the interval $(a,b)$
            \item $f(x) < f(a)$ for some point $x$ within the interval $(a,b)$
        \end{enumerate}
        \paragraph{Case 1.)} If $f(x)=K$, where $K$ is some constant, then we know that for all $x \in (a,b)$, $f'(x) = 0$. Therefore for any point $c \in (a,b)$, $f'(c) = 0$.
        \paragraph{Case 2.)} Since $f$ is a continuous function on the closed interval $(a,b)$ then by the Extreme Value Theorem we know that there is a maximum value within $[a,b]$. Additionally it is known that for all $f(x)$ that $f(x)>f(a)=f(b)$ meaning that a maximum value can not occur at either endpoint. \\
        Because the maximum must be inside of the interval $[a,b]$ we can say that at some point $ x = c$ where $c$ is a local maximum, that $f'(c) =0$
        \paragraph{Case 3.)} Since $f$ is a continuous function on the closed interval $(a,b)$, then by the Extreme Value Theorem we know that there exists a minimum value within $[a,b]$. Since there exists an $x$ such that $f(x) < f(a) = f(b)$ we can say that this minimum value can not occur at either endpoint.\\
        Because the minimum be be inside of the interval $(a,b)$ we can say that at some point $x = c$ where $c$ is a local minimum, that $f'(c) = 0$
    \end{proof}
\section{The Mean Value Theorem and Proof} \label{sect:MVT}
    \begin{proof}
        Consider a secant line that connects the points $(a,f(a))$ and $(b,f(b))$ the slope of that secant line is,
        $$\frac{f(b)-f(a)}{b-a}$$.
        By consider the point $(a,f(a))$, the equation of the line passing through the point in "point slope form" would be,
        $$y=f(a) +\frac{f(b)-f(a)}{b-a}(x-a)$$
        Let the function $g(x)$ be the vertical distance between the function $f$ and the equation of the secant line connecting $(a,f(a))$ and $(b,f(b))$.
        $$g(x) = f(x) -\left[f(a) +\frac{f(b)-f(a)}{b-a}(x-a)\right]$$
    \begin{figure}[H]
        \centering
        \includegraphics[width=0.5\linewidth]{Ch4_MVT_Fig6_v2.png}
        \caption{$g(x)$ is the difference between the secant line and $f(x)$}
        \label{fig:mvtDifference}
    \end{figure}
        We know that the function $g(x)$ is the difference between two continuous functions, so we can say that $g(x)$ itself is continuous on the interval $[a,b]$.\\
        Additionally since $g(x)$ is the difference between two differentiable functions, we can say that $g(x)$ itself is differentiable on the interval $[a,b]$.\\\\
        Since the secant line intersects the function $f$ at $x=a$ and $x-b$, (the endpoints), we can say that the vertical difference between the two points is zero. Therefore, $$0 = g(a) =g(b)$$
        We now see that the function $g(x)$ satisfies all three conditions to apply Rolle's Theorem, meaning that there exists a point $c\in(a,b)$ such that $g'(c)=0$.
        By differentiating $g'(c)$ we yield,
        $$g'(c)=f'(c)-\frac{f(b)-f(a)}{b-a}=0$$
        where we can now conclude,
        $$f'(c)=\frac{f(b)-f(a)}{b-a}$$
        
    \end{proof}
\begin{thebibliography}{100}
       \bibitem{Libre Texts} Libre Texts Mathematics "4.2: The Mean Value Theorem". Access Date 17 October 2025.\\Link to source - \href{https://math.libretexts.org/Bookshelves/Calculus/Map%3A_Calculus__Early_Transcendentals_(Stewart)/04%3A_Applications_of_Differentiation/4.02%3A_The_Mean_Value_Theorem}{(see here)}
       \bibitem{Wikipedia} Wikipedia. "The Mean Value Theorem". Last Modified 21 August 2025.\\Link to source - 
       \href{https://en.wikipedia.org/wiki/Mean_value_theorem}{(see here)}
       \bibitem{Emory} Emory - Oxford College , "Proof of the Mean Value Theorem". Access Date 17 October 2025.\\Link to source - \href{https://math.oxford.emory.edu/site/math111/proofs/meanValueTheorem/}{(see here)}
       \bibitem{Wikimedia} Wikimedia Commons - Rolles Theorem and Mean Value Theorem \href{fig:MVT}{See figure 2} \href{fig:rolles}{See figure 1}
       
\end{thebibliography}
\end{document}
